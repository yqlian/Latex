%----------------------------------------------------------------------------------------
%	PACKAGES AND THEMES
%----------------------------------------------------------------------------------------

\let\raggedright\relax

\documentclass{beamer}

\mode<presentation>{
% The Beamer class comes with a number of default slide themes
% which change the colors and layouts of slides. Below this is a list
% of all the themes, uncomment each in turn to see what they look like.

%\usetheme{default}
%\usetheme{AnnArbor}
%\usetheme{Antibes}
%\usetheme{Bergen}
%\usetheme{Berkeley}
%\usetheme{Berlin}
%\usetheme{Boadilla}
%\usetheme{CambridgeUS}
%\usetheme{Copenhagen}
%\usetheme{Darmstadt}
%\usetheme{Dresden}
%\usetheme{Frankfurt}
%\usetheme{Goettingen}
%\usetheme{Hannover}
%\usetheme{Ilmenau}
%\usetheme{JuanLesPins}
%\usetheme{Luebeck}
\usetheme{Madrid}
%\usetheme{Malmoe}
%\usetheme{Marburg}
%\usetheme{Montpellier}
%\usetheme{PaloAlto}
%\usetheme{Pittsburgh}
%\usetheme{Rochester}
%\usetheme{Singapore}
%\usetheme{Szeged}
%\usetheme{Warsaw}

% As well as themes, the Beamer class has a number of color themes
% for any slide theme. Uncomment each of these in turn to see how it
% changes the colors of your current slide theme.

%\usecolortheme{albatross}
%\usecolortheme{beaver}
%\usecolortheme{beetle}
%\usecolortheme{crane}
%\usecolortheme{dolphin}
%\usecolortheme{dove}
%\usecolortheme{fly}
%\usecolortheme{lily}
%\usecolortheme{orchid}
%\usecolortheme{rose}
%\usecolortheme{seagull}
%\usecolortheme{seahorse}
%\usecolortheme{whale}
\usecolortheme{wolverine}
}

\usefonttheme{professionalfonts}

\usepackage{graphicx} % Allows including images
\usepackage{booktabs} % Allows the use of \toprule, \midrule and \bottomrule in tables
\usepackage{amsmath}
\usepackage{xcolor}
\hypersetup{
  unicode={true},
  bookmarksopen={true},
  pdfborder={0 0 0},
  citecolor=blue,
  linkcolor=blue,
  anchorcolor=blue,
  urlcolor=blue,
  colorlinks=true,
  pdfborder=000
}
\setbeamertemplate{theorems}[numbered]
\setbeamertemplate{caption}[numbered]
\everydisplay{\color{red}}
\setbeamercovered{transparent}
\beamerdefaultoverlayspecification{<+->}

\AtBeginSection[]
{
\begin{frame}
\frametitle{Overview}
\tableofcontents[currentsection]
\end{frame}
}
\AtBeginSubsection[]
{
\begin{frame}
\frametitle{Overview}
\tableofcontents[sectionstyle=show/shaded,subsectionstyle=show/shaded]
\end{frame}
}

%----------------------------------------------------------------------------------------
%	TITLE PAGE
%----------------------------------------------------------------------------------------

\title[Short Title]{Full Title} % The short title appears at the bottom of every slide, the full title is only on the title page

\author{Name} % Your name
\institute[Short Institution] % Your institution as it will appear on the bottom of every slide, may be shorthand to save space
{
Full Institution \\ % Your institution for the title page
\medskip
\textcolor{blue}{yqlian.rol@gmail.com} % Your email address
}
\date{\today}


\begin{document}

\begin{frame}
\titlepage % Print the title page as the first slide
\thispagestyle{empty}
\addtocounter{framenumber}{-1}
\end{frame}

\begin{frame}
\frametitle{Overview} % Table of contents slide, comment this block out to remove it
\tableofcontents % Throughout your presentation, if you choose to use \section{} and \subsection{} commands, these will automatically be printed on this slide as an overview of your presentation
\end{frame}

%----------------------------------------------------------------------------------------
%	PRESENTATION SLIDES
%----------------------------------------------------------------------------------------

%------------------------------------------------
\section{Introduction} % Sections can be created in order to organize your presentation into discrete blocks, all sections and subsections are automatically printed in the table of contents as an overview of the talk
%------------------------------------------------

\begin{frame}
\frametitle{Introduction}
\begin{itemize}
\item Introduction 1
\item Introduction 2
\end{itemize}
\end{frame}

%------------------------------------------------

\section{Illustrations and Comments}

%------------------------------------------------

\begin{frame}
\frametitle{Illustrations and Comments}
\begin{itemize}
\item Illustrations and Comments
\end{itemize}
\end{frame}

%------------------------------------------------

\begin{frame}
\Huge{\centerline{The End}}
\end{frame}

%----------------------------------------------------------------------------------------

\end{document}
