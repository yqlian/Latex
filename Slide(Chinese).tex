\documentclass{beamer}

\mode<presentation>{
% The Beamer class comes with a number of default slide themes
% which change the colors and layouts of slides. Below this is a list
% of all the themes, uncomment each in turn to see what they look like.

%\usetheme{default}
%\usetheme{AnnArbor}
%\usetheme{Antibes}
%\usetheme{Bergen}
%\usetheme{Berkeley}
%\usetheme{Berlin}
%\usetheme{Boadilla}
%\usetheme{CambridgeUS}
%\usetheme{Copenhagen}
%\usetheme{Darmstadt}
%\usetheme{Dresden}
%\usetheme{Frankfurt}
%\usetheme{Goettingen}
%\usetheme{Hannover}
%\usetheme{Ilmenau}
%\usetheme{JuanLesPins}
%\usetheme{Luebeck}
\usetheme{Madrid}
%\usetheme{Malmoe}
%\usetheme{Marburg}
%\usetheme{Montpellier}
%\usetheme{PaloAlto}
%\usetheme{Pittsburgh}
%\usetheme{Rochester}
%\usetheme{Singapore}
%\usetheme{Szeged}
%\usetheme{Warsaw}

% As well as themes, the Beamer class has a number of color themes
% for any slide theme. Uncomment each of these in turn to see how it
% changes the colors of your current slide theme.

%\usecolortheme{albatross}
%\usecolortheme{beaver}
%\usecolortheme{beetle}
%\usecolortheme{crane}
%\usecolortheme{dolphin}
%\usecolortheme{dove}
%\usecolortheme{fly}
%\usecolortheme{lily}
%\usecolortheme{orchid}
%\usecolortheme{rose}
%\usecolortheme{seagull}
%\usecolortheme{seahorse}
%\usecolortheme{whale}
\usecolortheme{wolverine}
}

\usefonttheme{professionalfonts}

\usepackage[space,noindent]{ctex}
\usepackage{graphicx} % Allows including images
\usepackage{booktabs} % Allows the use of \toprule, \midrule and \bottomrule in tables
\usepackage{amsmath}
\usepackage{xcolor}
\hypersetup{
  unicode={true}, % 使用unicode来编码PDF字符串
  bookmarksopen={true}, % 展开书签
  pdfborder={0 0 0}, % 链接无框
  citecolor=blue,
  linkcolor=blue, % blue
  anchorcolor=blue,
  urlcolor=blue,
  colorlinks=true, % 注释掉此项则交叉引用为彩色边框(将colorlinks和pdfborder同时注释掉)
  pdfborder=000 % 注释掉此项则交叉引用为彩色边框
}

\setbeamertemplate{theorems}[numbered]
\setbeamertemplate{caption}[numbered]
\renewcommand\tablename{\color{blue}表}
\renewcommand\figurename{\color{blue}图}
\newtheorem{mythl}{\heiti 引理}
\newtheorem{mytht}{\heiti 定理}
\newtheorem{mythr}{\heiti 注}[section]
\newtheorem{mythc}{\heiti 推论}[section]
\newtheorem{mythd}{\heiti 定义}
\newtheorem{mytha}{\heiti 公理}
\newtheorem{mythp}{\heiti 命题}
\newtheorem{mythe}{\heiti 练习}
\newtheorem{myli}{\hei 例}[section]

\everydisplay{\color{red}}
\setbeamercovered{transparent}
\beamerdefaultoverlayspecification{<+->}

\AtBeginSection[]
{
\begin{frame}
\frametitle{报告提纲}
\tableofcontents[currentsection]
\end{frame}
}
\AtBeginSubsection[]
{
\begin{frame}
\frametitle{报告提纲}
\tableofcontents[sectionstyle=show/shaded,subsectionstyle=show/shaded]
\end{frame}
}


\begin{document}
\hypersetup{CJKbookmarks=true}

\title[短标题]{标题} % The short title appears at the bottom of every slide, the full title is only on the title page

\author{作者} % Your name
\institute[机构缩写] % Your institution as it will appear on the bottom of every slide, may be shorthand to save space
{
机构 \\ % Your institution for the title page
\medskip
\textcolor{blue}{\textit{yqlian.rol@gmail.com}} % Your email address
}
\date{\today} % Date, can be changed to a custom date


\begin{frame}
\titlepage % Print the title page as the first slide
\thispagestyle{empty}
\addtocounter{framenumber}{-1}
\end{frame}

\begin{frame}
\frametitle{报告提纲} % Table of contents slide, comment this block out to remove it
\tableofcontents % Throughout your presentation, if you choose to use \section{} and \subsection{} commands, these will automatically be printed on this slide as an overview of your presentation
\end{frame}

\section{基本概念}

\begin{frame}
\frametitle{基本概念}
\begin{itemize}
\item 1983年美国IEEE计算机学会对“软件可靠性”定义:
\begin{enumerate}
    \item 在规定条件下,在规定时间内,软件不引起系统失效的概率,该概率是系统输入和系统使用的函数,也是软件中存在的错误的函数;系统输入将确定是否会遇到已存在的错误;
    \item 在规定的时间周期内,在所述条件下执行所要求的功能的能力。
\end{enumerate}
\end{itemize}
\begin{itemize}
\item 设$F(t)$为软件故障概率函数,密度函数为$f(t)$,{\bf 可靠度函数}
\begin{equation}
R(t)=1-F(t)=P(\xi > t), \label{RelFun}
\end{equation}
随机变量$\xi$表示软件从运行开始到失效所经历的时间。
\end{itemize}
\end{frame}

\begin{frame}
\frametitle{基本概念}
\begin{itemize}
\item 软件{\bf 失效率}(Software Failure Rate)
\begin{equation}
\lambda(t)=\frac{f(t)}{R(t)}. \label{lambda}
\end{equation}
有时又称为{\bf 故障率}或{\bf 风险率}。
\end{itemize}
\begin{itemize}
\item 参数$\theta$的后验分布
\begin{equation}
\pi(\theta |x)=\frac{h(x,\theta)}{m(x)}=\frac{p(x|\theta)\pi(\theta)}{\int_\Theta p(x|\theta)\pi(\theta){\rm d}\theta}. \label{BayesF}
\end{equation}
由于$m(x)$是与$\theta$无关的因子,由分布的核的概念得
\begin{equation}
\pi(\theta |x)\propto p(x|\theta)\pi(\theta). \label{PiCore}
\end{equation}
\end{itemize}
\end{frame}

\section{实例分析}

\begin{frame}
\frametitle{实例分析}
\begin{itemize}
\item 实例分析
\end{itemize}
\end{frame}

\begin{frame}
\Huge{\centerline{The End}\centerline{Q \& A}}
\end{frame}

\end{document}
